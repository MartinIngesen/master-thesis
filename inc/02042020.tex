\chapter{Definitions/Terms}
\label{chap:definitions}

\todo{This is a todo note}
\\
\n{This is a regular note}
\\
\com{This is a supervisor comment}


\subsubsection{Context} -
\subsubsection{Context Engine} The part of the program that handles context.
\subsubsection{Event} A single Windows Event Log event.
\subsubsection{Goroutine} A lightweight thread handled by the Go runtime.
\subsubsection{Log line} A single line of a dataset, which consist of a single event.
\subsubsection{Sysmon} A Windows service and driver that uses XML configuration files to enhance the log data given by Windows Event Log.
\subsubsection{Windows Event Log} A built-in mechanism in Windows that logs telemetry of what happens on the system. Can be enhanced by installing and configuring Sysmon on the system.
\subsubsection{Worker} -



\chapter{Experiments}
\label{chap:experiments}

\section{Terminology}
\subsection{Context}
\subsubsection{Context Engine}

The context engine is 

\subsection{Worker}

A worker is a Goroutine that has the sole job of reading an event from a channel, trying to match the event to the given set of rules. If there is a match, the worker will decide if it has to 

\subsubsection{Goroutines}

A Goroutine is a lightweight thread that is managed by the Go runtime. it is not a thread in the traditional sense.


